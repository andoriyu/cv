\documentclass[11pt,a4paper]{moderncv}

%moderncvthemes are 'blue' (default), 'orange', 'red', 'green', 'grey' and 'roman'
% (for roman fonts, instead of sans serif fonts)
\moderncvtheme[orange]{classic}

\usepackage[utf8]{inputenc}

\usepackage[scale=0.8]{geometry}
% if you want to change the width of the column with the dates
%\setlength{\hintscolumnwidth}{3cm}

% only for the classic theme, if you want to change the width
% of your name placeholder (to leave more space for your address details
%\AtBeginDocument{\setlength{\maketitlenamewidth}{6cm}}

% required when changes are made to page layout lengths
\AtBeginDocument{\recomputelengths}

% personal data
\firstname{Serge}
\familyname{Ziryukin}
%\title{Resumé title (optional)}
\mobile{+37529-1822764}
%\phone{phone (optional)}
%\fax{fax (optional)}
\email{ftrvxmtrx@gmail.com}
\homepage{github.com/ftrvxmtrx}
%\photo[64pt]{picture}
%\quote{24 years old, married.}

% to show numerical labels in the bibliography; only useful if you make citations in your resume
\makeatletter
\renewcommand*{\bibliographyitemlabel}{\@biblabel{\arabic{enumiv}}}
\makeatother

% bibliography with mutiple entries
%\usepackage{multibib}
%\newcites{book,misc}{{Books},{Others}}

%\nopagenumbers{} % uncomment to suppress automatic page numbering for CVs longer than one page
%----------------------------------------------------------------------------------

\begin{document}
\maketitle

\section{Education}
\cventry{2003--2008}{B.S.}{Belarusian State University of Informatics and Radioelectronics}{Minsk}{\textit{Computers, Systems and Networks}}{}

\section{BS thesis}
\cvline{title}{\emph{Microkernel operating system based on L4 specification.}}
\cvline{supervisors}{Glecevich I.I.}
\cvline{description}{\small Two main types of kernels (monolithic and ${\mu}$) are described and compared, showing ${\mu}$-kernel drawbacks and benefits.
  L4-based ${\mu}$-kernel implementation for x86 and MIPS is provided, its loading/execution process and APIs are described.}

\section{Most used development tools}
\cvcomputer{Languages}{C, OCaml, Erlang, C++, Shell}{}{}
\cvcomputer{Op. systems}{Arch Linux, Mac OS X, FreeBSD}{}{}
\cvcomputer{Documentation}{Doxygen, ocamldoc, \LaTeX}{}{}
\cvcomputer{Build systems}{CMake, GNU Make, qmake}{}{}
\cvcomputer{SCM}{Git, Mercurial, SVN}{}{}
\cvcomputer{IDE}{Emacs, XCode}{}{}
\cvcomputer{WM}{xmonad}{}{}

\section{Languages}
%\cvlanguage{Russian}{Excellent}{}
\cvlanguage{English}{Intermediate}{}

\section{Experience}
\subsection{Vocational}

\cventry{2011--}{Software Engineer}{PowerMeMobile}{Minsk}{}{Telecom billing.\newline{}%
\begin{itemize}%
\item Built on Erlang/OTP.
\end{itemize}}
\newline

\cventry{2010}{Software Engineer}{Synesis}{Minsk}{}{HAL for NEC devboard.\newline{}%
\begin{itemize}%
\item In-kernel \href{http://en.wikipedia.org/wiki/Real-time_Transport_Protocol}{RTP/RTCP} support implementation;
\item Tests in Python.
\end{itemize}}
\newline

\cventry{2010}{Software Engineer}{Synesis}{Minsk}{}{Porting DVB-T software to NEC devboard.\newline{}%
\begin{itemize}%
\item Porting U-Boot to the devboard;
\item Porting Linux to the devboard;
\item MIPS binaries reverse-engineering;
\item Porting DVB-T software and platform-specific bugfixing;
\item Minimizing and putting the whole firmware into 8Mb flash device;
\item CMake build system.
\end{itemize}}
\newline

\cventry{2010--2011}{Software Engineer}{Synesis}{Minsk}{}{iPhone VoIP app (\small\url{http://viber.com}).\newline{}%
\begin{itemize}%
\item Bugfixing;
\item Helper tools for developers.
\end{itemize}}
\newline

\cventry{2008--2010}{Software Engineer}{Synesis}{Minsk}{}{iPhone game development.\newline{}%
\begin{itemize}%
\item Porting existing games to the iPhone platform.
\end{itemize}}
\newline

\cventry{2008}{Software Engineer}{Synesis}{Minsk}{}{OpenGLES-based UI tech demo for Freescale i.MX31 PDK.\newline{}%
\begin{itemize}%
\item The tech demo.
\end{itemize}}
\newline

\cventry{2008}{Software Engineer}{Synesis}{Minsk}{}{Porting previous project to Fujitsu board.\newline{}%
\begin{itemize}%
\item Cross-compiler toolchain;
\item Providing a way to run and test the software;
\item Adding Fujitsu board (with proprietary RTOS running on it) support to middleware.
\end{itemize}}
\newline

\cventry{2008}{Software Engineer}{Synesis}{Minsk}{}{DVB-T set-top box cross-platform software for three different boards.\newline{}%
\begin{itemize}%
\item Cross-platform middleware API design and implementation;
\item TI-DM6446 (with Linux running on it) specific middleware implementation;
\item Build system.
\end{itemize}}
\newline

\cventry{2007--2008}{Software Engineer}{Synesis}{Minsk}{}{DVB-S Linux-based set-top box (ST7109) software.\newline{}%
\begin{itemize}%
\item Tests in Python;
\item Extending automatic testing framework;
\item Bug fixing;
\item Screens;
\item Automatic software update functionality and tools.
\end{itemize}}
\newline
\\
\\
\subsection{Miscellaneous}
\cventry{2010--}{Sole developer}{}{}{}{Single cross-platform Quake I/II/III game engine.\newline \small\url{https://github.com/ftrvxmtrx/metaquake}\newline{}%
\begin{itemize}%
\item Cross-platform;
\item Nice modular architecture;
\item Clean and well-documented internal API;
\item Simplification through the code generation using OCaml;
\item A lot of tests.
\end{itemize}}
\newline

\cventry{2010--}{Sole developer}{}{}{}{QuakeC bytecode to native code library compiler.\newline \small\url{https://github.com/ftrvxmtrx/qc2lib}\newline{}%
\begin{itemize}%
\item \href{http://caml.inria.fr/ocaml/index.en.html}{OCaml};
\item Well-documented code;
\item \href{http://llvm.org/}{LLVM} usage in the future.
\end{itemize}}
\newline

\cventry{2010}{Sole developer}{}{}{}{Tool for finding unnecessary include directives.\newline \small\url{https://github.com/ftrvxmtrx/inclean}\newline}
\cventry{2010}{Sole developer with few contributions from others}{}{}{}{2D puzzle for Maemo-based Nokia N900.\newline \small\url{http://maemo.org/downloads/product/Maemo5/colorflood}\newline{}%
\begin{itemize}%
\item \href{http://qt.nokia.com}{Qt, C++};
\item CMake build system.
\end{itemize}}
\newline

\cventry{2009}{Sole developer with few contributions from others}{}{}{}{Flac/Ape/Wavepack + cue sheet into tracks splitter.\newline \small\url{http://split2flac.googlecode.com}\newline{}%
\begin{itemize}%
\item One small (652 LOC) POSIX-compiant shell script to do everything.
\end{itemize}}
\newline

\cventry{2009}{Sole developer}{}{}{}{Open-source 2D game engine.\newline \small\url{https://github.com/ftrvxmtrx/erszebet}\newline{}%
\begin{itemize}%
\item Lua for game code;
\item Chipmunk for physics;
\item Linux, FreeBSD, Mac OS X and iPhone support.
\end{itemize}}
\newline

\cventry{2005--2006}{Sole developer}{}{}{}{Client-side QuakeC implementation for Darkplaces game engine.\newline{}%
\begin{itemize}%
\item Heavily used by \small{\href{http://en.wikipedia.org/wiki/Nexuiz}{Nexuiz}.}
\end{itemize}}
\newline

\cventry{2005--2006}{Sole developer}{}{}{}{TomazQuake derived game engine.\newline{}%
\begin{itemize}%
\item Custom Quake-based network protocol to lower the traffic;
\item Rendering optimizations for high framerates.
\end{itemize}}
\newline

\cventry{2004--2005}{Sole developer, mapper, modeller}{}{}{}{Quake UT-like game mod.\newline}

\section{Interests}
\cvline{Programming}{\small \url{http://github.com/ftrvxmtrx} \newline \url{http://ohloh.net/accounts/ftrvxmtrx}}
\cvline{Problem solving}{\small \url{http://projecteuler.net/profile/ftrvxmtrx.png} \newline \url{http://spoj.pl/users/ftrvxmtrx}}
\cvline{Photography}{\small \url{http://picasaweb.google.com/ftrvxmtrx}}
\cvline{Music}{\small \url{http://last.fm/user/i515i}}
\cvline{Games}{\small Quake I/II/III, UT99, Scrabble, Warcraft II, etc}

\section{Personal information}
\cvline{Age}{24}
\cvline{Marital status}{married}
\cvline{Kids}{no}

\end{document}
