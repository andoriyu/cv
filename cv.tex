%% start of file `template.tex'.
%% Copyright 2006-2010 Xavier Danaux (xdanaux@gmail.com).
%
% This work may be distributed and/or modified under the
% conditions of the LaTeX Project Public License version 1.3c,
% available at http://www.latex-project.org/lppl/.


\documentclass[11pt,a4paper]{moderncv}

% moderncv themes
%\moderncvtheme[orange]{casual}                 % optional argument are 'blue' (default), 'orange', 'red', 'green', 'grey' and 'roman' (for roman fonts, instead of sans serif fonts)
\moderncvtheme[orange]{classic}                % idem

% character encoding
\usepackage[utf8]{inputenc}                   % replace by the encoding you are using

% adjust the page margins
\usepackage[scale=0.8]{geometry}
%\setlength{\hintscolumnwidth}{3cm}						% if you want to change the width of the column with the dates
%\AtBeginDocument{\setlength{\maketitlenamewidth}{6cm}}  % only for the classic theme, if you want to change the width of your name placeholder (to leave more space for your address details
\AtBeginDocument{\recomputelengths}                     % required when changes are made to page layout lengths

% personal data
\firstname{Serge}
\familyname{Ziryukin}
%\title{Resumé title (optional)}               % optional, remove the line if not wanted
%\address{Parnikovaya, 11/63}{220000}    % optional, remove the line if not wanted
\mobile{+37529-1822764}                    % optional, remove the line if not wanted
%\phone{phone (optional)}                      % optional, remove the line if not wanted
%\fax{fax (optional)}                          % optional, remove the line if not wanted
\email{ftrvxmtrx@gmail.com}                   % optional, remove the line if not wanted
\homepage{github.com/ftrvxmtrx}        % optional, remove the line if not wanted
%\extrainfo{jabber: ftrvxmtrx@gmail.com} % optional, remove the line if not wanted
\photo[64pt]{picture}                         % '64pt' is the height the picture must be resized to and 'picture' is the name of the picture file; optional, remove the line if not wanted
%\quote{Some quote (optional)}                 % optional, remove the line if not wanted

% to show numerical labels in the bibliography; only useful if you make citations in your resume
\makeatletter
\renewcommand*{\bibliographyitemlabel}{\@biblabel{\arabic{enumiv}}}
\makeatother

% bibliography with mutiple entries
%\usepackage{multibib}
%\newcites{book,misc}{{Books},{Others}}

%\nopagenumbers{}                             % uncomment to suppress automatic page numbering for CVs longer than one page
%----------------------------------------------------------------------------------
%            content
%----------------------------------------------------------------------------------
\begin{document}
\maketitle

\section{Most used tools for development}
\cvcomputer{Languages}{C, OCaml, Shell, C++, ObjC, Lua}{}{}
\cvcomputer{Op. systems}{Arch Linux, FreeBSD, Mac OS X}{}{}
\cvcomputer{Documentation}{Doxygen, ocamldoc, \LaTeX}{}{}
\cvcomputer{Build systems}{CMake, GNU Make, qmake}{}{}
\cvcomputer{SCM}{Git, Mercurial, SVN}{}{}
\cvcomputer{IDE}{Emacs, XCode}{}{}
\cvcomputer{WM}{xmonad}{}{}

\section{Languages}
%\cvlanguage{Russian}{Excellent}{}
\cvlanguage{English}{Intermediate+}{}

\section{Experience}
\subsection{Vocational}

\cventry{2010--}{Software Engineer}{Synesis}{Minsk}{}{Middleware for NEC devboard.\newline{}%
\begin{itemize}%
\item In-kernel RTP/RTCP media support.
\end{itemize}}
\newline

\cventry{2010}{Software Engineer}{Synesis}{Minsk}{}{Porting DVB-T software to NEC devboard.\newline{}%
\begin{itemize}%
\item Porting U-Boot to the devboard;
\item Porting Linux to the devboard;
\item Porting DVB-T software and platform-specific bugfixing;
\item Minimizing and putting the whole firmware into 8Mb flash device;
\item CMake build system.
\end{itemize}}
\newline

\cventry{2008--2010}{Software Engineer}{Synesis}{Minsk}{}{iPhone game development.\newline{}%
\begin{itemize}%
\item Porting existing games to the iPhone platform.
\end{itemize}}
\newline

\cventry{2008}{Software Engineer}{Synesis}{Minsk}{}{OpenGLES-based UI tech demo for Freescale i.MX31 PDK.\newline{}%
\begin{itemize}%
\item The tech demo.
\end{itemize}}
\newline

\cventry{2008}{Software Engineer}{Synesis}{Minsk}{}{Porting previous project to Fujitsu board.\newline{}%
\begin{itemize}%
\item Cross-compiler toolchain;
\item Providing a way to run and test the software;
\item Adding Fujitsu board support to middleware.
\end{itemize}}
\newline

\cventry{2008}{Software Engineer}{Synesis}{Minsk}{}{DVB-T set-top box cross-platform software for three different boards.\newline{}%
\begin{itemize}%
\item Cross-platform middleware API design and implementation;
\item TI-DM64446 specific middleware implementation.
\end{itemize}}
\newline

\cventry{2007--2008}{Software Engineer}{Synesis}{Minsk}{}{DVB-S set-top box (ST7109) software.\newline{}%
\begin{itemize}%
\item Tests in Python;
\item Extending automatic testing framework;
\item Bug fixing;
\item Screens;
\item Automatic software update functionality and tools.
\end{itemize}}
\newline

\subsection{Miscellaneous}
\cventry{2010--}{Sole developer}{}{}{}{QuakeC bytecode to native code library compiler.\newline \small\url{https://github.com/ftrvxmtrx/qc2lib}\newline}
\cventry{2010--}{Sole developer}{}{}{}{Single cross-platform Quake I/II/III game engine.\newline \small\url{https://github.com/ftrvxmtrx/metaquake}\newline}
\cventry{2010}{Sole developer}{}{}{}{Tool for finding unnecessary include directives.\newline \small\url{https://github.com/ftrvxmtrx/inclean}\newline}
\cventry{2010}{Sole developer}{}{}{}{2D puzzle for Maemo-based Nokia N900.\newline \small\url{http://maemo.org/downloads/product/Maemo5/colorflood}\newline}
\cventry{2009}{Sole developer}{}{}{}{Flac/Ape/Wavepack + cue sheet into tracks splitter.\newline \small\url{http://split2flac.googlecode.com}}\newline{}%
\cventry{2009}{Sole developer}{}{}{}{Open-source 2D game engine.\newline \small\url{https://github.com/ftrvxmtrx/erszebet}\newline{}%
\begin{itemize}%
\item Lua for game code;
\item Chipmunk for physics;
\item Linux, FreeBSD, Mac OS X and iPhone support.
\end{itemize}}
\newline

\cventry{2005--2006}{Sole developer}{}{}{}{Client-side QuakeC implementation for Darkplaces game engine.\newline}
\newline
\cventry{2005--2006}{Sole developer}{}{}{}{TomazQuake derived game engine with a lot of additional features.\newline}
\newline
\cventry{2004--2005}{Sole developer, mapper, modeller}{}{}{}{Quake UT-like game mod.\newline}

\section{Education}
\cventry{2003--2008}{B.S.}{Belarusian State University of Informatics and Radioelectronics}{Minsk}{\textit{Computers, Systems and Networks}}{}  % arguments 3 to 6 can be left empty

\section{BS thesis}
\cvline{title}{\emph{Microkernel operating system based on L4 specification.}}
\cvline{supervisors}{Glecevich I.I.}
\cvline{description}{\small Two main types of kernels (monolithic and ${\mu}$) are described and compared, showing ${\mu}$-kernel drawbacks and benefits.
  L4-based ${\mu}$-kernel implementation for x86 and MIPS is provided, its loading/execution process and APIs are described.}

\section{Interests}
\cvline{Programming}{\small \url{http://github.com/ftrvxmtrx} \newline \url{http://ohloh.net/accounts/ftrvxmtrx}}
\cvline{Problems}{\small \url{http://projecteuler.net/profile/ftrvxmtrx.png} \newline \url{http://spoj.pl/users/ftrvxmtrx}}
\cvline{Photography}{\small \url{http://picasaweb.google.com/ftrvxmtrx}}
\cvline{Music}{\small \url{http://last.fm/user/i515i}}
\cvline{Games}{\small Quake I/II/III, UT99, Scrabble, Warcraft II, etc}

\end{document}


%% end of file `template_en.tex'.
